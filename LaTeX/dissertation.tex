%% abtex2-modelo-trabalho-academico.tex, v-1.9.2 laurocesar
%% Copyright 2012-2014 by abnTeX2 group at http://abntex2.googlecode.com/ 
%%
%% This work may be distributed and/or modified under the
%% conditions of the LaTeX Project Public License, either version 1.3
%% of this license or (at your option) any later version.
%% The latest version of this license is in
%%   http://www.latex-project.org/lppl.txt
%% and version 1.3 or later is part of all distributions of LaTeX
%% version 2005/12/01 or later.
%%
%% This work has the LPPL maintenance status `maintained'.
%% 
%% The Current Maintainer of this work is the abnTeX2 team, led
%% by Lauro C�sar Araujo. Further information are available on 
%% http://abntex2.googlecode.com/
%%
%% This work consists of the files abntex2-modelo-trabalho-academico.tex,
%% abntex2-modelo-include-comandos and abntex2-modelo-references.bib
%%

% ------------------------------------------------------------------------
% ------------------------------------------------------------------------
% abnTeX2: Modelo de Trabalho Academico (tese de doutorado, dissertacao de
% mestrado e trabalhos monograficos em geral) em conformidade com 
% ABNT NBR 14724:2011: Informacao e documentacao - Trabalhos academicos -
% Apresentacao
% ------------------------------------------------------------------------
% ------------------------------------------------------------------------

\documentclass[en]{ufpethesis}

\usepackage{graphicx}			% Inclus�o de gr�ficos
\usepackage{caption}
\usepackage{subcaption}
\usepackage[table,xcdraw]{xcolor}
\usepackage{tabularx,ragged2e}
\usepackage{algorithm}% http://ctan.org/pkg/algorithms
\usepackage{algpseudocode}% http://ctan.org/pkg/algorithmicx
\newcolumntype{Y}{>{\raggedright\arraybackslash}X} % right "X" column
\newcolumntype{Z}{>{\raggedleft\arraybackslash}X} % left "X" column
\usepackage{sectsty}
\usepackage{ textcomp }
\usepackage[utf8]{inputenc}
\usepackage{setspace}
\usepackage{listings}
\usepackage{float}
\usepackage[T1]{fontenc}
\sectionfont{\bfseries\Large\raggedright}

\renewcommand{\baselinestretch}{1.5} %espa�amento de linhas
\setlength{\parindent}{1.5cm}

\let\Algorithm\algorithm
\renewcommand\algorithm[1][]{\Algorithm[#1]\setstretch{1}}
\algrenewcommand{\algorithmiccomment}[1]{\hfill\small$\triangleright$#1}
\renewcommand{\algorithmicforall}{\textbf{for each}}

%% Identifica��o:

% Universidade
% e.g. \university{Universidade de Campinas}
% Na UFPE, comente a linha a seguir
\university{Universidade Federal de Pernambuco}

% Endere�o (cidade)
% e.g. \address{Campinas}
% Na UFPE, comente a linha a seguir
\address{Recife}

% Instituto ou Centro Acad�mico
% e.g. \institute{Centro de Ci�ncias Exatas e da Natureza}
% Comente se n�o se aplicar
\institute{Centro de Inform\'{a}tica}

% Departamento acad�mico
% e.g. \department{Departamento de Inform�tica}
% Comente se n�o se aplicar
\department{Departamento de Inform\'{a}tica}

% Programa de p�s-gradua��o
% e.g. \program{P�s-gradua��o em Ci�ncia da Computa��o}
\program{P\'{o}s-gradua\c{c}\~{a}o em Ci\^{e}ncia da Computa\c{c}\~{a}o}

% �rea de titula��o
% e.g. \majorfield{Ci�ncia da Computa��o}
\majorfield{Ci\^{e}ncia da Computa\c{c}\~{a}o}

% T�tulo da disserta��o/tese
% e.g. \title{Sobre a conjectura $P=NP$}
\title{Using OLAP Queries for Data Analysis on Graph Databases}

% Data da defesa
% e.g. \date{19 de fevereiro de 2003}
\date{01 de setembro de 2017}

% Autor
% e.g. \author{Jos� da Silva}
\author{Nicolle Chaves Cysneiros}

% Orientador(a)
% Op��o: [f] - para orientador do sexo feminino
% e.g. \adviser[f]{Profa. Dra. Maria Santos}
\adviser[f]{Prof\textordfeminine. Dr\textordfeminine. Ana Carolina Salgado}

% ----
% In�cio do documento
% ----
\begin{document}

\frontmatter
\frontpage
\presentationpage

\begin{dedicatory}
Ao G e E do meu GEN
\end{dedicatory}

\acknowledgements
Agrade\c{c}o aos meus pais, por terem me apoiado durante todo o per\'{i}odo desse projeto e por terem batalhado durante toda a minha vida para me proporcionar oportunidades incr\'{i}veis como esse Mestrado. Agrade\c{c}o \`{a} Professora Carol, que atuou como minha mentora desde minha gradua\c{c}\~{a}o, me orientando e me ajudando nos caminhos que me trouxeram at\'{e} aqui. Agrade\c{c}o tamb\'{e}m aos meus amigos Luiz, Tomaz e todo o time Labcodes por terem me escutado e me dado for\c{c}as durante os momentos mais dif\'{i}ceis dessa jornada.

\resumo
Bancos de Dados (BDs) em Grafo s\~{a}o uma alternativa aos tradicionais BDs Relacionais e permitem uma melhor escalabilidade do sistema, al\'{e}m de uma maneira mais natural de representar dados altamente conectados. Os BDs em Grafo tamb\'{e}m permitem diferentes tipos de an\'{a}lises em grafos, como medidas de centralidade e algoritmos de detec\c{c}\~{a}o de comunidades. Apesar disso, ainda n\~{a}o existem ferramentas dispon\'{i}veis no mercado para fazer an\'{a}lise multidimensional em grafos, como os sistemas OLAP existentes que operam sobre BDs Relacionais. No meio acad\^{e}mico, existem algumas propostas de frameworks que visam a constru\c{c}\~{a}o de um cubo multidimensional composto por grafos agregados, obtidos a partir da combina\c{c}\~{a}o de n\'{o}s e arestas do grafo original de acordo com as dimens\~{o}es e medidas analisadas. Contudo, a maior parte das pesquisas s\~{a}o voltadas para a an\'{a}lise de grafos homog\^{e}neos, enquanto os trabalhos que se dedicam a grafos heterog\^{e}neos realizam a an\'{a}lise multidimensional a partir de um modelo intermedi\'{a}rio do dado original. Esse projeto prop\~{o}e um sistema para a realiza\c{c}\~{a}o de consultas OLAP em um Banco de Dados em Grafo sem a necessidade da gera\c{c}\~{a}o de um modelo intermedi\'{a}rio de dados para realizar an\'{a}lise em grafos heterog\^{e}neos. O sistema proposto \'{e} capaz de responder consultas OLAP a partir de grafos agregados extra\'{i}dos do grafo original, al\'{e}m de tamb\'{e}m realizar an\'{a}lises acerca da topologia do grafo. Neste trabalho s\~{a}o apresentados experimentos mostrando a efic\'{a}cia do sistema para responder \`{a}s consultas anal\'{i}ticas e compara\c{c}\~{o}es espec\'{i}ficas entre o sistema descrito e as solu\c{c}\~{o}es existentes.
% Palavras-chave do resumo em Portugu�s
\begin{keywords}
OLAP, Bando de Dados em Grafo, Grafos, An\'{a}lise de Dados
\end{keywords}

\abstract
Graph Databases (GDB) are an alternative to traditional Relational Databases and allow a better scalability for the system, in addition to representing highly connected data in a more natural way. GDBs also support different kind of network analysis, such as centrality measures and community detection algorithms. Despite this, there are still no tools available in the market for multidimensional analysis in graphs, such as existing OLAP systems that operate on Relational DBs. In the academic field, there are some framework proposals that aim at the construction of a multidimensional cube composed by aggregate graphs, which are obtained from the combination of vertices and edges of the original graph, according to the dimensions and measures being analysed. However, most part of the researches in this area are focused on the OLAP analysis for homogeneous graphs, while the works dedicated to heterogeneous graphs require an intermediate data model in order to execute the multidimensional analysis. This project proposes a system to execute OLAP queries in a Graph Database without the need to generate an intermediate data model to do multidimensional analysis on heterogeneous graphs. The proposed system is able to answer OLAP queries using aggregate graphs obtained from the original graph, as well as execute analysis about the topology of the graph. In this work, we present experiments showing the effectiveness of the system to answer the analytical queries and some qualitative comparisons between the proposed system and existing solutions.
% Palavras-chave do resumo em Portugu�s
\begin{keywords}
OLAP, Graph Databases, Graphs, Data Analysis
\end{keywords}

\tableofcontents
\listoffigures
% ----------------------------------------------------------
% ELEMENTOS TEXTUAIS
% ----------------------------------------------------------
\mainmatter

% ---
% Introduction
% ---
\chapter{Introduction}
In this chapter, we will present the motivations for the realisation of this work and give a clear definition of the problem to be addressed. The general and specific objectives of this research will be listed, as well as the remaining structure of the document.

\section{Motivation}
In recent years, our ability to collect data from different sources has increased significantly \cite{fan2013mining}. We can retrieve data from different devices, with different formats and different levels of connection. However, our capability in store, process and analyse these large collections of connected data has still room for improvement.

For this reason, Graph Databases (GDBs) have been gaining attention in the database community due to the good performance when dealing with highly connected data. In comparison to Relational Databases, where the execution performance of a query that requires intensive join operations deteriorates proportionally to data size, Graph Databases performance remains constant with respect to the size of the graph \cite{Robinson2015}. GDBs allow a more natural way to represent data as vertices and edges. Social networks, semantic web pages and recommendation systems are some examples of applications which handle data relationships and could perform better if the data were stored in a GDB \cite{Miller2013}. 

GDBs also provide a flexible data model, where the main information stored is the relationship between entities. This feature allows network processing based analysis to be done, such as pattern detection, edge path analysis and clustering techniques. These different analysis techniques allow the development of solutions for challenging problems, not only those approaches usually applied in traditional relational databases or data warehousing \cite{Loshin2013}. 

Our main motivation comes from the fact that there are no consolidate tools that can execute both network and multidimensional analysis on a Graph Database. In the academic field, there are some framework proposals that aim at the construction of a multidimensional cube composed by aggregate graphs, which are obtained from the combination of vertices and edges of the original graph, according to the dimensions and measures being analysed. However, most part of the researches in this area are focused on the OLAP analysis for homogeneous graphs, while the works dedicated to heterogeneous graphs require an intermediate data model in order to execute the multidimensional analysis.

\section{Problem Definition}
Given the scenario described in the previous section, we investigate the problem to be addressed from the question: ``How can we execute both network and multidimensional analysis on heterogeneous graphs data without the need to generate an intermediate data model?''. Considering the question proposed, the problem can be defined as: given a graph $G=(V,E)$, with a set of vertices $V$ and a set of edges $E$, what is the architecture and how is the operation of a system that is able to execute network analysis algorithms and OLAP queries over the graph $G$, without generating an intermediate data model.

\section{Objectives}
The general objective of this work is to build a system that supports the execution of network and multidimensional analysis on a Graph Database, without the generation of an intermediate data model for the graph. In order to achieve the general objective, some specific objectives were considered:
\begin{itemize}
\item Generate multidimensional view of the original data without changing the data model
\item Give support for OLAP operations to be executed over the multidimensional view
\item Give support for network analysis algorithms to be executed over the original data
\item Define the architecture of the complete system
\end{itemize}

\section{Expected Contributions}
Once the main problem and the objectives are defined, we expect from this work the following contributions:

\begin{itemize}
\item Overview of the state of the art for OLAP systems with Graph Databases
\item Architecture, specification and implementation of a Data Analysis System for Graph Databases.
\item Implementation of OLAP operators and network analysis algorithms, providing a comprehensive analysis of the graph data.
\item Experiments and qualitative analysis in comparison with existing frameworks.
\end{itemize}


\section{Document Structure}
The rest of the document is organized as follows:
\begin{description}
\item[Chapter 2] Introduces the main concepts related to the theoretical foundation of this work, such as OLAP systems, Graphs and Graph Databases.
\item[Chapter 3] Gives an overview of the state of the art for OLAP system with Graph Databases, presenting the main frameworks proposed in this area.
\item[Chapter 4] Presents the specification of the system proposed, detailing the architecture and the main components of the solution implemented.
\item[Chapter 5] Shows how the system was implemented and describes some experiments and the obtained results.
\item[Chapter 6] Concludes the document, recapitulating the work presented and giving insights for future work.
\end{description}

% ---
% Fundamentals
% ---
\include{chapters/2_background}

% ---
% Literature Revision
% ---
\chapter{Graph Cubes: State of the Art}
% ---

This chapter presents the most important research works published in the area of OLAP systems implemented using graph databases. The first work presented here dates back to 2011 and introduces several concepts - such as graph cube, aggregate graphs and cuboid queries - that serve as basis for other papers published since then.

\section{Graph Cube: On Warehousing and OLAP Multidimensional Networks}

One of the main works in graph analysis using OLAP methods is described in \cite{Zhao2011}, and it  introduces several concepts that will be used by other authors to describe their work in this area. Graph Cube is one of those concepts and it is defined as a multidimensional model that extends multidimensional networks to provide decision support features. A multidimensional network is a graph $N=(V,E,A)$, where $V$ is a set of vertices, $E$ is a set of edges and $A$ is a set of vertex-specific attributes. Each vertex in the graph is a multidimensional tuple and the attributes of the vertex define the graph cube dimensions.
 
A graph cube is formed by all possible aggregate networks calculated from the original multidimensional network. An aggregate network (often called cuboid) is a summarisation of the original graph with respect to one or more dimensions, which is calculated by applying an aggregate function (e.g. COUNT, SUM, AVERAGE, among others) on the vertices attributes. Consider the multidimensional network illustrated in Figure \ref{fig:figure12}.

\begin{figure}[ht]
\centering
\includegraphics[width=1\textwidth]{../multidimensional_graph.png}
\caption{Multidimensional network \cite{Zhao2011}}
\label{fig:figure12}
\end{figure}

\begin{figure}[ht]
\centering
\includegraphics[width=0.8\textwidth]{../aggregate_graph.png}
\caption{Aggregate Network by \emph{Gender} dimension \cite{Zhao2011}}
\label{fig:figure13}
\end{figure}

The vertices of the graph presented in Figure \ref{fig:figure12} represent individuals and the edges represent the relationship between these individuals. The table on the right side of Figure \ref{fig:figure12} describes the attributes of each vertex: an unique \emph{ID}, \emph{Gender}, \emph{Location}, \emph{Profession} and \emph{Income}. Figure \ref{fig:figure13} shows the cuboid obtained by applying the function COUNT on the attribute \emph{Gender}: two vertices represent the possible values for the attribute (Male and Female) and they contain the number of vertices in the original graph with the respective \emph{Gender} value. It is important to notice that the relationship between individuals was also aggregated in the resulting network, e.g. the aggregate network show that there are 9 relationships between male and female individuals in the original graph, which is represented by an edge with weight of 9.

The dimension of a cuboid is given by the set of non-aggregate dimensions of the cuboid. For instance, the cuboid in Figure \ref{fig:figure13} has the dimension $\{Gender\}$. A cuboid $A'$ is an ancestor of another cuboid $A"$ if $dim(A')$ contains $dim(A'')$. Given these definitions, all possible cuboids of a graph cube can be organised in a graph cube lattice, ordering the cuboids according to its ancestors. A multidimensional network $N$ with $n$ dimensions has $2^n$ cuboids in the graph cube. Figure \ref{fig:figure14} shows the graph cube lattice for the multidimensional network introduced in Figure \ref{fig:figure12}, considering only \emph{Gender}, \emph{Location} and \emph{Profession} as dimensions.

\begin{figure}[ht]
\centering
\includegraphics[width=0.6\textwidth]{../graph_lattice.png}
\caption{Graph Cube Lattice \cite{Zhao2011}}
\label{fig:figure14}
\end{figure}

The paper proposes two OLAP query models:
\begin{description}
\item[Cuboid Query] Aggregate vertices and edges based on the dimension requested in the query and can work with any aggregate function (SUM, AVG, etc). For instance, consider a graph with vertices containing the attributes (\emph{Gender}, \emph{Location}, \emph{Profession}). A cuboid query for this graph can be $(Gender, *, *)$ which will result in an aggregate graph showing all the vertices with the same Gender value aggregated and the edges also aggregate by the function COUNT.
\item[Crossboid Query] Return the aggregate network between two or more cuboids structures. An example of a crossboid query can be the aggregate network between an user with $ID = 3$ and all the locations.
\end{description}

Given that the size of the graph cube lattice is exponential with respect to the number of dimensions of the original multidimensional network, the paper proposes a partial materialisation in order to process queries. The partial materialisation is implemented using a greedy algorithm that selects k cuboids $(k < 2^n)$ to be materialised according to the benefit of those cuboids to improve the cost for query evaluation.
 
The graph cube implementation described above is also studied by other works. In \cite{Denis2013}, a distributive approach is proposed using map reduce jobs to calculate each step of the aggregation process. In \cite{Khan2014}, a new algorithm to compute the Graph Cube (iGraphCubing) is proposed, using a new prunning method based on the Structural Significance measure.This measure takes into account three factors:
 
\begin{itemize}
\item The diversity of the attribute value in the neighborhood for each vertex
\item The clustering coefficient
\item The density around each vertex
\end{itemize}

The work in \cite{Zhao2011} presents experiments evaluating the effectiveness and the efficiency of the method proposed. The effectiveness was evaluated by analysing OLAP queries submitted to the graph cube. The efficiency was evaluated by analysing the graph cube performance depending if it was fully or partially materialised in disk. In conclusion, the paper proposes an implementation of a graph cube obtained from an homogeneous attributed graph, but it does not consider heterogeneous networks neither attributed edges.

\section{Graph OLAP: Towards Online Analytical Processing on Graphs}

The Graph OLAP framework  proposed in \cite{Chen2008} takes as input a set of network snapshots, where each snapshot contains a graph and a set of informational attributes that describes the snapshot. In this scenario, the paper defines two dimension types: informational dimensions (formed by the informational attributes) and topological dimensions (formed by the attributes of the vertices and edges of the graph). The authors distinguish different semantics for OLAP operations in graphs, so these operations are categorised into Informational OLAP (I-OLAP) and Topological OLAP (T-OLAP).

The framework gives as output an aggregate graph with a summarised view of the snapshots set. The type of the aggregate graph returned by the framework can also be categorised in Informational Aggregate Graph (I-aggregate graph) and Topological Aggregate Graph (T-aggregate graph). 
 
Informational Aggregate Graph is computed based on a set of network snapshots that have informational dimensions with same values. Figure \ref{fig:figure15} shows an example of an i-aggregate graph composed by snapshots describing the co-author relationship between individual authors. The co-author relationship is grouped by a certain conference (\emph{sigmod}, \emph{vldb}, \emph{icde}), by a class of conferences (\emph{db-conf}), by a specific period of time (\emph{2004}, \emph{2005}) and by a group of time periods (\emph{all-years}). It is important to notice that the graph in Figure \ref{fig:figure15} shows different levels of aggregations for each co-author relationship. 

\begin{figure}[ht]
\centering
\includegraphics[width=0.6\textwidth]{../i_aggregated_graph_example.png}
\caption{Example of Informational Aggregate Graph \cite{Chen2008}}
\label{fig:figure15}
\end{figure}

Classic OLAP operations in an i-aggregate graph can be interpreted as follows:
\begin{description}
\item[Roll-up] Aggregate multiple snapshots to form a higher level summary of information
\item[Drill-down] Return to lower-level snapshots from aggregate graph
\item[Slice / dice] Select a subset of snapshots based on informational dimensions
\end{description}
 
Topological Aggregate Graph is obtained based on a single network, where the vertices are the result of applying the aggregate function to the vertices of the original network with the same attribute value. Figure \ref{fig:figure16} shows an example of a t-aggregate graph where the information about co-author relationship between individual authors in one snapshot was aggregated into co-author relationship between the institutions the authors belong to.

\begin{figure}[ht]
\centering
\includegraphics[width=0.7\textwidth]{../t_aggregated_graph_example.png}
\caption{Example of Topological Aggregate Graph \cite{Chen2008}}
\label{fig:figure16}
\end{figure}

Classic OLAP operations in an t-aggregate graph can be interpreted as follows:
\begin{description}
\item[Roll-up] Merge topological elements (vertices or edges) and replace them by corresponding higher-level elements
\item[Drill-down] Split merged elements into lower-level elements
\item[Slice / dice] Select a subgraph of a snapshot based on topological dimensions
\end{description}

The Topological OLAP is further explained in \cite{Qu2011}. This work takes into consideration two properties (T-Distributiveness and T-Monotonicity) used to classify how different measures can be performed in an OLAP Graph. A measure function is considered T-Distributive if the result of the function applied to high-level vertices from the graph can be obtained by the computation of pre-computed results of the same function applied to lower-level vertices from the same graph. On the other hand, a measure is considered T-Monotone, if the data search space can be pruned given a user-defined threshold, by dropping vertices pairs with measures that do not satisfy the threshold. The paper shows experiments using common constraint function, proven to be T-Distributives and/or T-Monotones, such as SUM, MIN, MAX, Density, Degree Centrality, Closeness Centrality, among others.

\section{HMGraph}

An Heterogeneous and Multidimensional Graph OLAP framework (HMGraph OLAP) is proposed in \cite{Yin2012}. This framework uses a graph model similar to Graph OLAP \cite{Chen2008}, but it adds the concept of Entity Dimensions due to the heterogeneity of the input graphs (Graph OLAP framework only handles homogeneous graphs). Figure \ref{fig:figure17} shows an example of a heterogeneous multidimensional network, highlighting the entity attributes of the graph.

\begin{figure}[ht]
\centering
\includegraphics[width=0.7\textwidth]{../heterogeneous_graph_example.png}
\caption{Example of a heterogeneous multidimensional network \cite{Yin2012}}
\label{fig:figure17}
\end{figure}

Entity attributes are the attributes that describe the characteristics of an entity. In the graph illustrated in Figure \ref{fig:figure17}, organ and age are entity attributes of author entity. Entity Dimension is related to the types of vertices in the graph.

Like Graph OLAP, HMGraph can perform I-OLAP and T-OLAP operations, but it can also perform rotate and stretch operations. The rotate operation is done by changing vertices into edges and edges into vertices, as shown in Figure  \ref{fig:figure18}. The stretch operation is done by changing edges into entities and adding edges between the recently created entity and the vertices previously connected to the transformed edge, as shown in Figure \ref{fig:figure19}.

\begin{figure}[ht]
\centering
\includegraphics[width=0.7\textwidth]{../rotate_operation_example.png}
\caption{Example of rotate operation \cite{Yin2012}}
\label{fig:figure18}
\end{figure}

\begin{figure}[ht]
\centering
\includegraphics[width=0.7\textwidth]{../stretch_operation_example.png}
\caption{Example of stretch operation \cite{Yin2012}}
\label{fig:figure19}
\end{figure}

Even though the work presented in \cite{Yin2012} draws attention to the importance of heterogeneous networks in real world application, the paper does not provide further implementation detail on how the framework can be used with real data.

\section{Pagrol: PArallel GRaph OLap over Large-scale Attributed Graphs}

The work presented in \cite{Wang2014} proposes a parallel Graph OLAP system, adopting the Hyper Graph Cube model that extends attributed graphs to support decision making services. The model proposed in this paper is similar to the Graph Cube described in \cite{Zhao2011}, with the main difference being the presence of attributes also in the graph edges. In this scenario, there are two types of dimensions: vertex dimensions and edge dimensions. Figure \ref{fig:figure20} shows an example of an attributed graph.

\begin{figure}[ht]
\centering
\includegraphics[width=0.8\textwidth]{../attributed_graph.png}
\caption{Example of attributed graph \cite{Wang2014}}
\label{fig:figure20}
\end{figure}

Given an attributed graph with n vertex dimensions and m edge dimensions, the Hyper Graph Cube will contain $2^{n+m}$ aggregate graphs obtained as described by the work of \cite{Zhao2011}. This Hyper Graph Cube can be seen as the cartesian product between all the vertex-aggregate networks (when one or more vertex dimensions are aggregated) and the edge-aggregate networks (when one or more edge dimensions are aggregated). This cube arrangement can support the following categories of queries:

\begin{description}
\item[Category 1] Queries answered by information stored either in a vertex or in an edge attributes. For example:``How many relationships appeared in 2012?'' or ``What is the percentage of users in each different profession in this network?''
\item[Category 2] Queries answered by integrating the knowledge stored at both vertex and edges attributes. For example: ``What is the trend of the number of relations appearing between USA and SG (Singapore) in the last 3 years?''
\item[Category 3] Queries answered by an aggregate graph, that provides a summarised view of the data along some dimensions. For example: ``What is the graph structure as grouped by users' gender as well as relationship type?''
\end{description}

The Hyper Graph Cube also supports roll-up and drill-down operations, along both vertex and edge dimensions. For instance, if we have an aggregate graph along Location dimension according to City value, we can roll up to obtain an aggregate graph according to State value.
 
The materialisation for the Hyper Graph Cube is done using Map-Reduce(MR) jobs. Since vertex and edges are stored in two different tables in the distributed file system (DFS), the materialisation is done in two steps: first the two tables are joined into one flat table containing all the dimensions and the second step performs the cube computation. This process is optimised using self-contained joins and batching techniques.

\section{GRAD Graph Cubes}

One of the most recent works in this area is presented in \cite{ghrab2015framework}. This paper proposes a new technique for extracting multidimensional concepts and building OLAP cubes from heterogeneous property graphs. The authors propose a new classification of graph measures based on the aggregation type and computation algorithm:
\begin{description}
\item[Content-based measure] Calculated based on graph's vertices and edges attributes. They are similar to traditional OLAP measures.
\item[Graph-based measure] Obtained by applying graph algorithms. They capture topological properties of the graph.
\item[Graph as measure] Different aggregation levels of a graph can be considered measures.
\end{description}

Given a property graph with two distinct classes of vertices, the authors explore candidate dimensions, measures and cubes that can be obtained from the graph. The example used throughout the paper is a movie graph: it has movie vertices linked to vertices representing the actors that acted in the movie, as shown in Figure \ref{fig:figure21}. The dimensions obtained by a subset of vertices and edges attributes are called inter-class dimensions.

\begin{figure}[ht]
\centering
\includegraphics[width=0.8\textwidth]{../movie_graph.png}
\caption{Original movie graph \cite{ghrab2015framework}}
\label{fig:figure21}
\end{figure}

Once the dimensions are selected, a graph lattice is defined by all possible OLAP aggregations obtained by aggregating the intra-class dimensions. The inter-class measures fall back in one of the aforementioned categories (content-based,  graph-specific or graph as measure).

Figure \ref{fig:figure22} shows the aggregate graph obtained by grouping movies by their release date and actors by their birth date and gender. The graph shows the average ranking and rating of the ACTS relationship between grouped actors and movies.

\begin{figure}[ht]
\centering
\includegraphics[width=1\textwidth]{../interclass_dimension.png}
\caption{Aggregate Graph for inter-class dimensions \cite{ghrab2015framework}}
\label{fig:figure22}
\end{figure}

This paper also proposes a technique for building OLAP cubes extracted from a graph modelled according to the analysis-oriented graph database model GRAD. This model provides advanced graph structures, integrity constraints and graph algebra. According to the authors, traditional property graphs only support OLAP analysis of inter-class dimensions, while additional capabilities brought by GRAD allows the analysis of information stored within each vertex.
 
The GRAD model consider heterogeneous, attributed, labelled graphs and supports complex type attributes on the vertices. This model introduces special analytical structures called hypernodes, that represent real world entities grouped by classes. Each hypernode is a subgraph formed by an entity vertex - which contains the label and the identifier attributes - attributes vertices - linked to the entity vertex and represent the non-identifier attribute - and literal vertices - which stores the effective value of its attribute vertex. Figure \ref{fig:figure23} shows an example of a movie graph modelled with GRAD.

\begin{figure}[!h]
\centering
\includegraphics[width=0.5\textwidth]{../grad_model.png}
\caption{Movie graph on GRAD model \cite{ghrab2015framework}}
\label{fig:figure23}
\end{figure}

Based on this model, the paper defines Intra-class Dimensions as a subset of attributes vertices. The Intra-class Measures are identified by the attribute vertex label and are calculated in a similar way as the measures in property graph model. Figure \ref{fig:figure24} shows the result of applying aggregation function to the original GRAD graph in order to calculate the revenue measure, aggregating the Location according to the Country Name (CN) attribute.

\begin{figure}[!h]
\centering
\includegraphics[width=1\textwidth]{../intraclass_dimension.png}
\caption{Aggregate Graph for intra-class dimensionl \cite{ghrab2015framework}}
\label{fig:figure24}
\end{figure}

This framework's implementation used Neo4J for the graph storage and HDFS  (Hadoop Distributed File System) for distributed processing. The architecture is also composed by a middleware layer that is responsible for computing the aggregate graph and measures, using GraphX\footnote{https://spark.apache.org/graphx/} library to calculate graph-specific measures.

\section{Comparative Analysis}

The first works done in OLAP analysis on graph focused on homogeneous graph datasets and defined ground concept of this area, such as aggregate graphs and graph lattice. Several operators were proposed and, in general, three types of measures were taken into consideration:
\begin{description}
\item[Informational / Content-based / Attribute-based measure] Similar to traditional OLAP measures, this information is obtained by applying an aggregate function on the vertex's attributes.
\item[Topological / Aggregate Graph measure]  This type of measure gives information about the topology of the graph and is obtained by applying aggregate function on vertices and edges, generating a graph as a measure.
\item[Graph-based / specific measure] This type of measure is based on graph analysis theory and can be represented by a number or a subgraph.
\end{description}

One relevant characteristic of the works presented so far is the little explanation given on how the framework was indeed implemented, which made their understanding rather difficult. The Table \ref{tb:table1} shows a comparison between all the frameworks presented in this chapter, regarding the type of graph, dimensions and operations supported by each one.

\begin{table}[!ht]
\setlength\extrarowheight{2pt}
\caption{Comparison of studied frameworks}
\label{tb:table1}
\begin{tabularx}{\textwidth}{|Y|Z|Z|Z|}
\hline
\cellcolor[HTML]{C0C0C0}\textbf{Framework} & \cellcolor[HTML]{C0C0C0}\textbf{Graph} & \cellcolor[HTML]{C0C0C0}\textbf{Dimensions} & \cellcolor[HTML]{C0C0C0}\textbf{Operations}\\\hline
{\cellcolor[HTML]{EFEFEF} Graph Cube} & Homogeneous & Vertex Attributes & Cuboid and Crossboid Query\\\hline
{\cellcolor[HTML]{EFEFEF} Graph OLAP} & Homogeneous & Informational and Topological & I-OLAP and T-OLAP Operations\\\hline
{\cellcolor[HTML]{EFEFEF} HMGraph} & Heterogeneous & Informational, Topological and Entity & I-OLAP, T-OLAP, Rotate and Stretch Operations\\\hline
{\cellcolor[HTML]{EFEFEF} Pagrol} & Homogeneous & Vertex and Edge Attributes & 3 Query Category and Roll-up/Drill-down Operations \\ \hline
{\cellcolor[HTML]{EFEFEF} GRAD Graph Cubes} & Heterogeneous & Inter-class and Intra-class & - \\ \hline
\end{tabularx}
\end{table}

The work proposed here will be focused in heterogeneous graph datasets and will support the three types of measures described previously, in a similar way as presented by the work on GRAD Graph Cubes. In addition to that, this work will also explore OLAP operations and network analysis on graph databases without the need to define a new graph model, as suggested by previous works, eliminating the extra step of parsing operational data to the new model.

\section{Final Considerations}

In this chapter, we presented the main research works published related to OLAP system using Graph Databases. The majority of the works available only supported homogeneous graph, but they introduced important concepts of the area, such as graph cubes and aggregate graph. The implementations that actually gave support to heterogeneous graph, proposed different graph models in order to answer analytical queries. In the next chapter, we will specify a simple OLAP system using Graph Database without the need to define a new graph model.


% ---
% Specification
% ---
\chapter{OLAP Analysis on Graph Database}
In this chapter we will propose a system capable of executing OLAP analysis and that supports heterogeneous graphs, without the need to define a new graph data model. Initially, we will contextualise the proposed system and introduce a running example that will be used to better explain the system operation. Then, the system's architecture is presented and its main components are further detailed in the following sections.

\section{Contextualisation}

In Chapter 3, we investigated some of the main works in the area of Graph OLAP. Most of them only give support to homogeneous graphs \cite{Zhao2011}\cite{Chen2008}\cite{Wang2014}, while real world graph-like data contains different types of vertices and edges.The frameworks HMGraph \cite{Yin2012} and GRAD Graph Cube \cite{ghrab2015framework} support heterogeneous graphs, but they propose a new multidimensional model in order to do OLAP analysis.

The objective of the system proposed here is to support OLAP analysis on heterogeneous graph databases without the need to re-model operational data. This will be done by adding a layer of pre-processed aggregate graphs and an analytical query processor module on top of the operational graph database.

\section{Running Example}
Consider the Database System and Logic Programming (DBLP) dataset as the running example that will be used throughout this chapter to help explaining the main concepts of the system proposed. The DBLP\footnote{http://dblp.uni-trier.de/faq/What+is+dblp.html} is an online computer science bibliography that, up until May 2016, indexes more than 3.3 million publications by more than 1.7 million authors. For this example, we will consider that the data was extracted and modelled according to the schema shown in Figure \ref{fig:figure25} and described as follows:

\begin{itemize}
\item Each publication becomes a vertex with label \emph{Publication} and with the attributes \emph{title}, \emph{year} and \emph{venue}.
\item Each author becomes a vertex with label \emph{Author} and with the attribute \emph{name}
\item Edges labeled \emph{PUBLISHED} connect Author vertices to Publication vertices, representing the relationship between an author and their published work.
\item Edges labeled \emph{COAUTHOR} connect Author vertices to other Author vertices, representing the relationship between authors that have contributed to the same published work.
\end{itemize}

\begin{figure}[ht]
\centering
\includegraphics[width=0.8\textwidth]{../dblp_schema_with_attr.png}
\caption{Schema representation of the DBLP data graph}
\label{fig:figure25}
\end{figure}

Figure \ref{fig:figure26} shows a subset of the original DBLP dataset, modelled according to the schema representation depicted in Figure \ref{fig:figure25}. The following graph will be used as our running example throughout this chapter.

\begin{figure}[ht]
\centering
\includegraphics[width=1\textwidth]{../running_example.png}
\caption{Running Example with subset of DBLP dataset}
\label{fig:figure26}
\end{figure}

\section{Dimensions and Measures}

As discussed in Chapter 2, an OLAP system is a tool that facilitates multidimensional analysis of the data. In order to perform such kind of analysis, it is necessary to define the dimensions and measures that will be considered during the multidimensional analysis:
\begin{description}
\item[Dimension] Given a labeled property graph $G = (V, E, L_V, L_E, A_V, A_E)$, where:
\begin{itemize}
\item $V$ is a set of vertices
\item $E \subseteq V \times V$ is a set of edges 
\item $L_V$ is a set of vertices labels and $L_E$ is a set of edge labels
\item $A_V = \{a_1, \dots, a_n\}$ is a set of vertex attributes, where $a_i = (k_i, m_i)$ is a key-value pair, $k_i$ is the attribute key and $m_i$ is the attribute value. Each vertex $v_i \in V$ is associated with a set of attributes. $A_E = \{b_1, \dots, b_n\}$ is the set of edge attributes defined in the same way as vertex attributes.
\end{itemize}
 A dimension is given by $d = (a, l)$, where $a \in A_V$ and $l \in L_V$. 

In our running example, we can define $d_1 = (journal, Publication)$ and $d_2 = (year, Publication)$, i.e. the \emph{Publication} attributes \emph{journal} and \emph{year} are dimensions $d_1$ and $d_2$ in our OLAP system, respectively.

\item[Measure] Given a labeled property graph $G = (V, E, L_V, L_E, A_V, A_E)$, a measure $m$ is a calculation computed over the graph $G$ using a function $F$ ($m=F(G)$), that can return the type of measures defined by  \cite{ghrab2015framework}:
\begin{description}
\item[Content-based measure] For this kind of measure is calculated based on the vertices and edges attributes and the function $F \in \{SUM, COUNT, AVG, or other aggregate functions\}$ used in the calculation are similar to the ones used in an OLAP system. 

In our running example, the total number of authors that published a work in 2007 is a content-based measure that is calculated using the function $COUNT$, which will count the number of \emph{Author} vertices that have a relationship \emph{PUBLISHED} to \emph{Publication} vertices that have attribute \emph{year} equals to 2007.
\item[Graph-based measure] This type of measure is calculated by applying a network analysis algorithm over the graph, i.e. a network analysis algorithm is used as the function $F$. 

In our running example, the list of authors that most contributed with other authors is a graph-based measure that is computed by applying the degree centrality measure to the \emph{Author} vertices of the graph.
\item[Graph as measure] This kind of measure is given by different aggregation levels of a graph and the function $F$ that calculates this measure is the aggregate function that will generate the aggregate graph. 

In our running example, the network of authors and  publications aggregated according to the venue in which the work was published in is a graph that represents a measure.
\end{description}
\end{description}

\section{Architecture}

The Graph OLAP system proposed in this work attempts to provide an efficient way to answer analytical queries without having to propose a new graph data model, which would imply changing the original data source model. The Figure \ref{fig:figure27} depicts the architecture of the system, illustrating its main components: Graph Aggregators, Aggregated Graphs and Analytical Query Processor.

The Graph Aggregators are modules that are responsible for processing the original data and generate Aggregate Graphs, which are stored in Aggregate Graph Databases. The Analytical Query Processor is in charge of processing the incoming query and the user will determine whether it should be answered by processing the original or the aggregate data, based on the type of measure being analysed.

\begin{figure}[ht]
\centering
\includegraphics[width=0.8\textwidth]{../Architecture.png}
\caption{OLAP Analysis over Graph Databases Architecture}
\label{fig:figure27}
\end{figure}

The system's input is an analytical query submitted by the user (1), that will be processed by the Analytical Query Processor (AQP) (2). According to the type of measure required by the user, the AQP will determine which specific processor will handle the query: graph-based, content-based or graph measures processors. If the user asks for a graph-based measure, the AQP will consume the original data stored in the graph database (3) to calculate the measure. On the other hand, if the user requires a content-based or a graph as measure, the AQP will calculate the measure based on the data from Aggregate Graphs (4), which are also stored in graph databases.

The Aggregate Graphs are generated by the Graph Aggregators (GAs) (5), which are defined during the design process of the system by the project designer. The project designer is responsible to define what are the dimensions considered in the OLAP system and, therefore, create the GAs that will generate all possible aggregate graphs for the dimensions. More details on Aggregate Graphs and Graph Aggregators are given in the following sections.

The data source considered for this system is a Graph Database that follows the labeled property graph model and supports heterogeneous graphs. This means that vertices can have one or more labels indicating different types of entities. Vertices and edges can have properties. We assume that the data stored in the GDB is integrated, i.e. the data in the repository is consistent, well-formatted and normalised.

\section{Aggregate Graph}
Given a graph $G = (V, E, L_V, L_E, A_V, A_E)$ and a set of dimensions $D = \{d_1, \dots, d_n\}$, where $d_i \in A_V \cup A_E$, an aggregate graph is generated by applying an aggregate function $F$ to one or more dimensions. The result is a new graph $G_A = (V_A, E_A, L_{VA}, L_{EA}, A_{VA}, A_{EA})$, where:
\begin{itemize}
\item $VA = \{v^A_1, \dots, v^A_n\}$ is a set of aggregate vertices, where each vertex $v^A_i$ either (i) corresponds to the result of applying the function $F$ to a set of vertices $V' \subseteq V$ that is associated with a set of attributes $\{a_1, \dots, a_k\}$ containing one or more dimensions in $D$ or (ii) corresponds to a vertex in $V$.
\item $E_A \subseteq V_A \times V_A$ is a set of aggregate edges, where each edge $e^A_i$ either (i) corresponds to the result of combining a set of edges $E' \subseteq E$ that connects one or more vertices in $V$ that were aggregated in $V_A$ or (ii) corresponds to an edge in $E$.
\item $L_{VA}$ is a set of aggregate vertex labels and $L_{EA}$ is a set of aggregate edge labels
\item $A_{VA} = \{a_1, \dots, a_n\}$ is the set of attributes for the aggregate vertices, where $a_i = (k_i, m_i)$ is a key-value pair, $k_i$ is the attribute key and $m_i$ is the attribute value. Each aggregate vertex $v_i \in V_A$ is associated with an attribute vector. $A_{EA} = \{b_1, \dots, b_n\}$ is the set of aggregate edge attributes defined in the same way as aggregate vertices attributes.
\end{itemize}

Consider the graph depicted in Figure \ref{fig:figure26} of our running example. Figure \ref{fig:figure28} shows the aggregate graph $G_A$ obtained by applying the aggregate function COUNT to the dimension set $D = \{d\}$, where $d = (year, Publication)$. Notice that the resulting aggregate graph ends up with the same \emph{Author} vertices as the original graph, since these vertices do not have attributes contained in the dimension set $D$. The \emph{Publication} vertices were aggregated according to their \emph{year} attribute, resulting in three vertices representing the works published in 2017, 2016 and 2015. The aggregate vertices also store the measure calculated using the function COUNT. The edges with the label PUBLISHED were combined, representing the connection between each author and the group of works published in a specific year. The combined edges also store the measure obtained by the use of COUNT function.

\begin{figure}[!h]
\centering
\includegraphics[width=0.4\textwidth]{../aggregate_graph_running_example.png}
\caption{Aggregate Graph obtained from running example graph}
\label{fig:figure28}
\end{figure}

\section{Graph Aggregators}

The Graph Aggregators (GAs) are modules responsible for generating the aggregate graph that will be used to answer the analytical query submitted by the user. During the design process of the system, the Project Designer is responsible for building the GAs based on the dimensions and measures the system should be able to analyse. Each GA will receive as input the original graph G stored in the Graph Database, the set of dimensions to be aggregated D, the aggregate function F and should provide as output an aggregate graph as defined in the previous section. Algorithm \ref{alg:algorithm1} describes the process performed by a GA in order to generate an aggregate graph.

\begin{algorithm}[!h]
 \caption{Graph Aggregator Process}\label{alg:algorithm1}
  \begin{algorithmic}[1]
    \Function{$generateAggGraph$}{$G, F, D$}
      \State $dimValue$ \Comment{set of values for dimensions being aggregated}
      \State $aggVertices$ \Comment{map from dimensions values to set of vertices with corresponding values}  \label{alg:line2}
      \State $aggEdges$ \Comment{map from dimensions values to set of edges with corresponding values} \label{alg:line3} 
      \State $nonAggVertices$ \Comment{set of vertices that were not aggregated}  \label{alg:line31} 
      \State $originalVertices \gets G.vertices$ \Comment{set of vertices from the original graph} \label{alg:line32}
      \ForAll{ $vertex \textrm{ in } originalVertices$}  \label{alg:line4}
        \If {$vertex  \textrm{ in } D$}  \label{alg:line5}
          \State $dimValue \gets getDimValue(vertex, D)$
           \If {$dimValue  \textrm{ in } aggVertices$}  \label{alg:line7}
             \State $aggregateVertices(dimValue, vertex, F)$  \label{alg:line8}
             \State $aggregateEdges(dimValue, vertex, F)$  \label{alg:line9}
           \Else
              \State $aggVertices.add(dimValue, vertex)$  \label{alg:line11}
              \State $aggregateEdges(dimValue, vertex, F)$  \label{alg:line12}
           \EndIf
       \Else 
          \State $nonAggVertices(vertex)$ \Comment{vertex does not contain the dimensions being aggregated}  \label{alg:line15}
        \EndIf
      \EndFor
      \State $aggGraph \gets (aggVertices \cup nonAggVertices, aggEdges)$  \label{alg:line18}
      \State \Return $aggGraph$  \label{alg:line19}
    \EndFunction
  \end{algorithmic}
\end{algorithm}

The GA algorithm starts by creating an empty set of aggregate vertices and edges (lines \ref{alg:line2} and \ref{alg:line3}) that will compose the final aggregate graph that will be returned. Then, the GA will iterate over all the vertices (nodes) of the original graph G (line \ref{alg:line4}), checking for each node if it has the same label as the dimension being aggregated (line \ref{alg:line5}). If the node has the same label, the algorithm checks if already exists an aggregate node for the dimension value of the node (line \ref{alg:line7}). If the aggregate node exists, we collapse the value of the node to the value of the aggregate node using the function F and updating the aggregate node value (line \ref{alg:line8}). We also aggregate the edges that are connected to the node accordingly (line \ref{alg:line9}). If the aggregate node does not exist, we add a new aggregate node with the initial value equals to node's dimension value and aggregate its edges accordingly (lines \ref{alg:line11} and \ref{alg:line12}). If the node does not have the same label as the dimension being aggregated, we only add the node as it is to the non aggregate vertices set (line \ref{alg:line15}). Finally, we setup the aggregate graph and return it (lines \ref{alg:line18} and \ref{alg:line19}).

Once the aggregate graph is generated, it will be stored in a graph database and it will be accessed by the Analytical Query Processor.

\section{Analytical Query Processor}

The Analytical Query Processor (AQP) is responsible for processing the query submitted by the user. This component contains three modules that will process the query differently based on the type of measure requested by the user:
\begin{description}
\item[Content-based measure] To calculate this measure, the AQP consumes the data from an aggregate graph and uses the aggregate function to give the resulting measure, which can be a single value, a list or a table of values. This module's response is similar to the response given by traditional OLAP systems.

From our running example, we can ask for the number of publications by year. In this case AQP consumes the data from the aggregate graph illustrated in Figure \ref{fig:figure28} and it would list the nodes with \emph{Publication Aggregate} label, which already contains the count measure as attribute.

\item[Graph-based measure] To calculate this measure, the AQP consumes the data from the original graph database and executes network analysis algorithms on the data. 

In our running example, we could submit a query asking for the \emph{Author} node with highest centrality degree. The AQP calculates the centrality degree for each node in the original graph using an external library. Then, it should return the node with id \textbf{A2}, since it is the Author node with the highest number of connections with other nodes.

\item[Graph as measure] For this type of measure, the AQP also consumes data from an aggregate graph, but in this case, the measure is the aggregated graph itself. Therefore, this module does not need to perform other calculations.

For instance, the aggregate graph shown in Figure \ref{fig:figure28} can be considered a measure if the user requests a topological view of the original data grouping the publications by year.
\end{description}

\section{Final Considerations}
In this chapter, we showed the main components of the proposed system and what is the general operation to answer an analytical query. We also specified in details how each one of the components works and what are their roles in the data analysis process. In the next chapter, we will report how the proposed system was implemented and show some experiments and the results obtained.

% ---
% Implementation and Experiments
% ---
\chapter{Implementation and Experiments}

In this chapter, we will show how the system specified in the Chapter 4 was implemented and what were the technologies used. We will also describe the experiments made and analyse the results obtained in comparison to existing solutions. Finally, we will discuss the difficulties found in the implementation and experimentation process.

\section{Used Technologies}
The Graph Aggregator (GA) algorithm was implemented using Python programming language, in version 2.7. The original data and the aggregate graphs were stored in a Neo4J database. In order to connect the GA to the Neo4J database, it was necessary to use the Python library Py2Neo, in version 3.1.2. The Analytical Query Processor uses the compiler built in Neo4J and the query accepted as input to the system is written using Cypher query language.

As mentioned in Chapter 2, Neo4J is the most popular graph database in the industry according to DB Engines website. In comparison with other DBs from the same category, Neo4J has a good performance considering time to process a query and the amount of memory consumed by the database.

The queries submitted to Neo4J are written in Cypher \cite{Neo4jCypher}, which is a declarative query language inspired by SQL and that describes graph patterns using ASCII characters. Figure \ref{fig:figure29} shows an example of how Cypher represents a relationship in the query syntax. This language also allows to create, update and delete vertices and edges. Since Cypher uses the terms ``nodes" and ``relationships" to refer to ``vertices'' and ``edges'' respectively, we will interchange these words accordingly throughout the text from now on.

\begin{figure}[ht]
\centering
\includegraphics[width=0.6\textwidth]{../cypher_pattern_simple.png}
\caption{Cypher syntax representation of a relationship in the graph \cite{Neo4jCypher}}
\label{fig:figure29}
\end{figure}

Neo4J query language also includes a series of clauses and expressions similar to SQL, such as  $WHERE$, $ORDER BY$, $LIMIT$, $AND$, among others. The Cypher query shown in Figure \ref{fig:figure30} is an example of the general syntax of the language and it shows how it is possible to restrict the results by a certain threshold using the clause $WHERE$. The mentioned query will return a subgraph containing the nodes with labels $Label1$ and $Label2$ that have a relationship of type $TYPE$ with a property value above a certain threshold.

\begin{figure}[ht]
\centering
\includegraphics[width=0.8\textwidth]{../cypher_general_syntax.png}
\caption{Example of Cypher Query syntax \cite{Neo4jCypher}}
\label{fig:figure30}
\end{figure}

The data stored in a Neo4J database can be accessed using a Java API or a REST API by default. In order to facilitate the access to the data using a Python application we used an external library called Py2Neo, which wraps REST API requests to execute Cypher commands on the database. In this way, it was possible to implement the algorithm for the Graph Aggregators, where it consumes the original data, generate an aggregate graph and stores it in another instance of a Neo4J database.

Another interesting feature in Neo4J is its web-based user interface. This interface provides a direct way to submit Cypher queries to the database and a visualisation of the results. Figure \ref{fig:figure31} is a screenshot of the interface, showing the result of a simple query submitted in the text area at the top of the screen. The submitted query returns a subgraph showing the relationship between the Author Felix Naumann with all his publications present in the database. The interface shows each node as a circle and relationship as an arrow, it also shows the properties and labels of nodes and relationships when they are clicked. The experiments shown in this chapter will be displayed using the graphic interface provided by Neo4J.

\begin{figure}[ht]
\centering
\includegraphics[width=1\textwidth]{../neo4j_user_interface.png}
\caption{Neo4J User Interface}
\label{fig:figure31}
\end{figure}

\section{Dataset}

The dataset used for the experiments was the Database System and Logic Programming (DBLP) computer science bibliography, available in (http://dblp.uni-trier.de/db/), which contains more than 3.8 million publications published by more than 1.7 million authors. The dataset can be downloaded as a XML file accompanied by a DTD file that describes the schema of the data. The Listing in \ref{lst:listing1} is an excerpt of the DBLP XML file showing how a publication is structured in the document.

\lstset{breaklines=true,
	caption={DBLP XML File Excerpt},
	label={lst:listing1},
	captionpos=b, 
	frame=single,
	morekeywords={article, author, title, pages, year, volume, journal, number, ee, url, mdate, key}}

\begin{lstlisting}
<article mdate="2017-01-03" key="journals/jacm/GanorKR16">
  <author>Anat Ganor</author>
  <author>Gillat Kol</author>
  <author>Ran Raz</author>
  <title>Exponential Separation of Information and Communication for Boolean Functions.</title>
  <pages>46:1-46:31</pages>
  <year>2016</year>
  <volume>63</volume>
  <journal>J. ACM</journal>
  <number>5</number>
  <ee>http://dl.acm.org/citation.cfm?id=2907939</ee>
  <url>db/journals/jacm/jacm63.html#GanorKR16</url>
</article>
\end{lstlisting}

The excerpt shown in Listing  \ref{lst:listing1} refers to an article published in J. ACM journal and it contains information about the article's authors, title, pages in journal, year of publication and other informations about the journal's volume, number and electronic edition location. Each publication also has a unique key and the date of the last modification as attributes and an element with the url of the publication in the DBLP website. Besides journal articles, the DBLP dataset also contains books, thesis, conferences and workshop papers, among others.

Once the XML file was downloaded, all the data was parsed and stored into a SQLite database. In order to have a more controlled environment for the experiments, we selected a subset of the original data, considering only papers and articles published in the following venues since 2014:

\begin{itemize}
\item SIGMOD International Conference on Management of Data (SIGMOD)
\item Brazilian Symposium on Databases (SBBD)
\item International Conference on Very Large Databases (VLDB)
\item IEEE International Conference on Data Engineering (ICDE)
\item World Wide Web: Internet and Web Information Systems (WWW)
\end{itemize}

The selected subset of publications was imported to a Neo4J instance, following the schema depicted in Figure \ref{fig:figure32}. Each publication became a node in the graph database, with three attributes: (i) the title; (ii) the year of publication and (iii) the acronym of the venue. The authors of each publication also became a node, uniquely identified by the name of the author, and they are connected with the publications node by a relationship of type PUBLISHED. Authors that have contributed in the same publication are also connected by a relationship of type CO\_AUTHORSHIP.

\begin{figure}[ht]
\centering
\includegraphics[width=0.6\textwidth]{../dblp_schema.png}
\caption{DBLP dataset schema in graph database}
\label{fig:figure32}
\end{figure}

By the end of the DBLP subset loading process to Neo4J, we had 887 Publication nodes, 2.398 Author nodes, 6.754 PUBLISHED relationships and 25.572 CO\_AUTHORSHIP relationships.

\section{Experiments and Results}

With the Neo4J database loaded with DBLP data, we executed the Graph Aggregator (GA) algorithm passing as parameters the dimensions year and venue of a Publication node and the COUNT aggregate function. Figure \ref{fig:figure33} shows a subgraph of the aggregate graph generated by the GA, with one aggregate node representing all the publications on ICDE 2016 and some of the authors that published on that conference, that year. From the measure attribute of the aggregate node, we know that ICDE had 60 publications in 2016.

\begin{figure}[!h]
\centering
\includegraphics[width=0.8\textwidth]{../aggregate_graph_subgraph.png}
\caption{Subgraph from the Aggregate Graph generated by the GA}
\label{fig:figure33}
\end{figure}

Once we generated the aggregate graph, we were able to do an experiment to evaluate the effectiveness of the proposed system. For that, we submitted queries to calculate the three types of supported measures: content-based, graph-specific and graph as measure.

\subsection{Content-based Measure}

For this type of measure, we submitted a query asking for the amount of publications by venue in the year of 2016. This query was submitted to the aggregate graph generated by the GA and it returns all the nodes with the label Aggregate\_Publication where the dimension year has the value 2016. The result is then ordered by the number of publications measure.

Figure \ref{fig:figure34} depicts the result of the query, listing the venues ordered by the amount of publications in the year of 2016. The conference SIGMOD appears at the top of the list with 61 publications for that year. In comparison to a traditional relational OLAP system, this query is corresponding to a slice operation, in which we slice a part of the venue dimension based on the value of the dimension year.

\begin{figure}[!h]
\centering
\includegraphics[width=0.8\textwidth]{../exp_content_measure.png}
\caption{Result of experiment with a content-based measure query}
\label{fig:figure34}
\end{figure}

\subsection{Graph-specific Measure}

In order to test this type of measure, we submitted two centrality measures queries to the original dataset in the graph database. The first measure calculated was the degree centrality of each Author node considering the CO\_AUTHORSHIP relationship. As mentioned in Chapter 2, the degree centrality is given by the number of adjacent node, i.e. the number of CO\_AUTHORSHIP relationship an Author node has.

Figure \ref{fig:figure35} shows the result of the centrality measure query, listing the 10 authors with the highest degree centrality in the original network. For our dataset, Michael J. Franklin is the author with the highest number of adjacent Author nodes, which means that he is one of the focal points in the network.

\begin{figure}[!h]
\centering
\includegraphics[width=0.8\textwidth]{../exp_degree_centrality.png}
\caption{Result of experiment with the degree centrality measure for Authors}
\label{fig:figure35}
\end{figure}

The second measure calculated was the betweenness centrality of the authors. This centrality measure is given by the frequency in which an author node appears in the shortest path of any other two author nodes in the graph. In Cypher, there is a built-in function to retrieve all shortest path between two nodes called $allShortestPaths$. Since getting all the shortest path between the combination of any 2 nodes in the graph is such a complex computation, we considered the paths with maximum 3 degrees of separation and we avoided inverse relationships by comparing the id of the nodes. Finally, we returned the 10 authors that most appeared in the list of the shortest paths.

Figure \ref{fig:figure36} shows the result of author's betweenness centrality calculation. Reaffirming his importance in the co-authorship network, Michael J. Franklin appears again in the first position of the list, meaning that he is an important point of collaboration between authors. One interesting result is related to the second place for both experiments: the second author with the highest degree centrality is not the same as the second author with the highest betweenness centrality. This means that, even so Volker Marki co-authored publications with more authors than Beng Chin Ooi, the latter is part of more collaboration chains between two other authors in the network.

\begin{figure}[!h]
\centering
\includegraphics[width=0.8\textwidth]{../exp_betweenness_centrality.png}
\caption{Result of experiment with the betweenness centrality measure for Authors}
\label{fig:figure36}
\end{figure}

\subsection{Graph as Measure}
This type of measure is given by the aggregate graph itself, since it is a representation of data aggregated by dimensions. For our running example, this measure represents the topological disposition of the relationship between publications and authors when the data is aggregated according to the dimensions year and venue.  

Figure \ref{fig:figure37} shows a screenshot of a subgraph retrieved from the aggregate graph. The subgraph was limited to 400 relationships in order to facilitate the visualisation. From the screenshot we can notice the topology distribution of author nodes that published works on ICDE in 2016 and 2015, specially the authors that published in both editions of the conference.

\begin{figure}[!h]
\centering
\includegraphics[width=1\textwidth]{../graph_measure.png}
\caption{Result of experiment with the graph as a measure}
\label{fig:figure37}
\end{figure}

\section{Result Analysis}
Given the results obtained from the experiments described in the previous section, we are able to confirm the effectiveness of the proposed system in supporting multidimensional analysis on graph database. The presented solution also supports the execution of graph-based analysis, such as centrality measures. In this way we are able to extract three different types of measures from the same origin data aggregated according to user defined dimensions.

In comparison to other works in this area, the proposed system gives full support for heterogeneous graphs without the need to change the original data schema. Focusing the comparison with the framework presented in \cite{Ghrab2013}, our system is able to answer the same types of query without the extra step to change the data model of our original dataset. Table \ref{tb:table2} summarises the comparison between existing frameworks and our proposed system.

\begin{table}[!ht]
\setlength\extrarowheight{2pt}
\caption{Comparison between existing frameworks and our proposed system}
\label{tb:table2}
\begin{tabularx}{\textwidth}{|Y|Z|Z|Z|}
\hline
\cellcolor[HTML]{C0C0C0}\textbf{Framework} & \cellcolor[HTML]{C0C0C0}\textbf{Graph} & \cellcolor[HTML]{C0C0C0}\textbf{Dimensions} & \cellcolor[HTML]{C0C0C0}\textbf{Operations}\\\hline
{\cellcolor[HTML]{EFEFEF} Graph Cube} & Homogeneous & Vertex Attributes & Cuboid and Crossboid Query\\\hline
{\cellcolor[HTML]{EFEFEF} Graph OLAP} & Homogeneous & Informational and Topological & I-OLAP and T-OLAP Operations\\\hline
{\cellcolor[HTML]{EFEFEF} HMGraph} & Heterogeneous & Informational, Topological and Entity & I-OLAP, T-OLAP, Rotate and Stretch Operations\\\hline
{\cellcolor[HTML]{EFEFEF} Pagrol} & Homogeneous & Vertex and Edge Attributes & 3 Query Category and Roll-up/Drill-down Operations \\ \hline
{\cellcolor[HTML]{EFEFEF} GRAD Graph Cubes} & Heterogeneous & Inter-class and Intra-class & - \\ \hline
{\cellcolor[HTML]{EFEFEF} \textbf{Using OLAP Queries for Data Analysis on Graph Databases}} & \textbf{Heterogeneous} & \textbf{Vertex and Edge Attributes} & \textbf{OLAP Operations and Network Analysis Algorithms} \\ \hline
\end{tabularx}
\end{table}

\section{Difficulties Found}

Unfortunately, we are unable to provide a more precise comparison between the framework in \cite{Ghrab2013} and our proposal due to the lack of experiment description. A similar issue also applies to other works presented in Chapter 3, i.e. the state of the art for Graph Cubes and OLAP analysis on graph databases.

The published papers in this area fail in providing enough description on how the proposed solution was implemented and where the dataset used in the experiments is available. Furthermore, when experiments are presented in the paper, they only compare different version of the same framework. Until now, it hasn't been proposed an experiment that can be replicated amongst different frameworks.

The absence of a benchmark for experiments in this area makes the evaluation of the proposed system difficult. In order to compare ourselves to others, we can only rely on qualitative measures, based on the features presented by existing frameworks.

Another difficulty found during the implementation of the system was the size of the graph supported by Neo4J. The DBLP has more than 3.8 million publications, but we limited our dataset according to some conferences and journals, as mentioned before. At first, we wanted to conduct the experiments using at least 5.000 publications, but we were not able to load all the nodes and relationships to Neo4J. Our script to load the data kept getting error related to the communication with Neo4J REST API and we couldn't figure out how to fix it. In order to be able to finish all the experiments on time, we had to reduce the number of publications loaded to the database.

\section{Final Considerations}
In this chapter, we detailed how the system proposed by this work was implemented, specifying what were the technologies used in the process. The dataset used in the experiments was presented as well as the results obtained. Finally, we presented a qualitative comparison with existing solutions and listed the main difficulties found during the system's implementation.

% ---
% Conclusion
% ---
\chapter{Conclusion}

In this dissertation, we addressed the issue of executing multidimensional and network analysis on a Graph Database. At first, the main concepts related to this issue were presented. Then, we had an overview of academic works in this area, giving a brief summary of each paper and comparing the frameworks they presented according to the following criteria: type of graph supported, OLAP dimensions and operations implemented. After this initial study, it was possible to notice that only two frameworks supported heterogeneous graphs, but they required the generation of an intermediate data model in order to execute OLAP analysis.

Once the state of the art for the area was covered, we specified the architecture and the main components of a data analysis system over Graph Databases, which supports heterogeneous graphs and the execution of OLAP queries and network analysis algorithms without the need of an intermediate data model. Next, we proceed to describe in details how the proposed system was implemented and what were the technologies used. Finally, some experiments were presented, as well as an analysis of the results obtained and a qualitative comparison between the system built and the existing frameworks. 

\section{Contributions}
As an outcome of the work presented in this document, we have the following contributions:
\begin{itemize}
\item Documented the state of the art for OLAP systems with Graph Databases and established an analytical comparison between existing frameworks, defining the main characteristics to be taken into consideration for the comparison.
\item Specification and implementation of a Data Analysis System for Graph Databases, describing how its main components operate and how to build such system using open-source tools (e.g. Python and Neo4J).
\item Implementation of OLAP operators and network analysis algorithms, providing a comprehensive analysis of the graph data. The execution of OLAP queries was possible due to a set of aggregate graphs that provided a multidimensional view of the graph data.
\item Definition of experiments and qualitative analysis in comparison with existing frameworks. All the difficulties found in this area regarding execution of experiments and quantitative comparison between solutions were described.
\end{itemize}

\section{Future Work}
The user interface of the proposed system relies on the interface provided by Neo4J. An interesting work that could be done is building a specific interface, where the user could execute OLAP queries and network analysis algorithm in a more friendly way. This specific interface could also provide different types of data visualisation depending on the type of measure the user requested.

Regarding the difficulties found during the implementation of the proposed system, an interesting issue that still needs to be tackled is the standardisation of the experiments. Amongst academic papers in this area, there is no consensus on how a comparative experiment should be done. Define general experiments and a benchmark dataset  would be a great contribution for this area, since it would allow a more precise quantitative comparison between existing solutions.


% ----------------------------------------------------------
% ELEMENTOS P�S-TEXTUAIS
% ----------------------------------------------------------
\backmatter
% ----------------------------------------------------------

% ----------------------------------------------------------
% Refer�ncias bibliogr�ficas
% ----------------------------------------------------------
\nocite{*}
\bibliographystyle{alpha}
\bibliography{references}

\end{document}
