\chapter{Conclusion}

In this dissertation, we addressed the issue of executing multidimensional and network analysis on a Graph Database. At first, the main concepts related to this issue were presented. Then, we had an overview of academic works in this area, giving a brief summary of each paper and comparing the frameworks they presented according to the following criteria: type of graph supported, OLAP dimensions and operations implemented. After this initial study, it was possible to notice that only two frameworks supported heterogeneous graphs, but they required the generation of an intermediate data model in order to execute OLAP analysis.

Once the state of the art for the area was covered, we specified the architecture and the main components of a data analysis system over Graph Databases, which supports heterogeneous graphs and the execution of OLAP queries and network analysis algorithms without the need of an intermediate data model. Next, we proceed to describe in details how the proposed system was implemented and what were the technologies used. Finally, some experiments were presented, as well as an analysis of the results obtained and a qualitative comparison between the system built and the existing frameworks. 

\section{Contributions}
As an outcome of the work presented in this document, we have the following contributions:
\begin{itemize}
\item Documented the state of the art for OLAP systems with Graph Databases and established an analytical comparison between existing frameworks, defining the main characteristics to be taken into consideration for the comparison.
\item Specification and implementation of a Data Analysis System for Graph Databases, describing how its main components operate and how to build such system using open-source tools (e.g. Python and Neo4J).
\item Implementation of OLAP operators and network analysis algorithms, providing a comprehensive analysis of the graph data. The execution of OLAP queries was possible due to a set of aggregate graphs that provided a multidimensional view of the graph data.
\item Definition of experiments and qualitative analysis in comparison with existing frameworks. All the difficulties found in this area regarding execution of experiments and quantitative comparison between solutions were described.
\end{itemize}

\section{Future Work}
The user interface of the proposed system relies on the interface provided by Neo4J. An interesting work that could be done is building a specific interface, where the user could execute OLAP queries and network analysis algorithm in a more friendly way. This specific interface could also provide different types of data visualisation depending on the type of measure the user requested.

Regarding the difficulties found during the implementation of the proposed system, an interesting issue that still needs to be tackled is the standardisation of the experiments. Amongst academic papers in this area, there is no consensus on how a comparative experiment should be done. Define general experiments and a benchmark dataset  would be a great contribution for this area, since it would allow a more precise quantitative comparison between existing solutions.