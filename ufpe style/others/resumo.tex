
% resumo em inglês
\begin{resumo}[Abstract]
\begin{otherlanguage*}{english}

%\noindent
Graph Databases (GDB) are an alternative to traditional Relational Databases and allow a better scalability for the system, in addition to representing highly connected data in a more natural way. GDBs also support different kind of network analysis, such as centrality measures and community detection algorithms. Despite this, there are still no tools available in the market for multidimensional analysis in graphs, such as existing OLAP systems that operate on Relational DBs. In the academic field, there are some framework proposals that aim at the construction of a multidimensional cube composed by aggregate graphs, which are obtained from the combination of vertices and edges of the original graph, according to the dimensions and measures being analysed. However, most part of the researches in this area are focused on the OLAP analysis for homogeneous graphs, while the works dedicated to heterogeneous graphs require an intermediate data model in order to execute the multidimensional analysis. This project proposes a system to execute OLAP queries in a Graph Database without the need to generate an intermediate data model to do multidimensional analysis on heterogeneous graphs. The proposed system is able to answer OLAP queries using aggregate graphs obtained from the original graph, as well as execute analysis about the topology of the graph. In this work, we present experiments showing the effectiveness of the system to answer the analytical queries and some qualitative comparisons between the proposed system and existing solutions.
   \vspace{\onelineskip}

   \noindent
   \textbf{Keywords}: OLAP; graph databases; graphs; data analysis.
 \end{otherlanguage*}
 \end{resumo}


% resumo em português
\begin{resumo}[Resumo]
Bancos de Dados (BDs) em Grafo s\~{a}o uma alternativa aos tradicionais BDs Relacionais e permitem uma melhor escalabilidade do sistema, al\'{e}m de uma maneira mais natural de representar dados altamente conectados. Os BDs em Grafo tamb\'{e}m permitem diferentes tipos de an\'{a}lises em grafos, como medidas de centralidade e algoritmos de detec\c{c}\~{a}o de comunidades. Apesar disso, ainda n\~{a}o existem ferramentas dispon\'{i}veis no mercado para fazer an\'{a}lise multidimensional em grafos, como os sistemas OLAP existentes que operam sobre BDs Relacionais. No meio acad\^{e}mico, existem algumas propostas de frameworks que visam a constru\c{c}\~{a}o de um cubo multidimensional composto por grafos agregados, obtidos a partir da combina\c{c}\~{a}o de n\'{o}s e arestas do grafo original de acordo com as dimens\~{o}es e medidas analisadas. Contudo, a maior parte das pesquisas s\~{a}o voltadas para a an\'{a}lise de grafos homog\^{e}neos, enquanto os trabalhos que se dedicam a grafos heterog\^{e}neos realizam a an\'{a}lise multidimensional a partir de um modelo intermedi\'{a}rio do dado original. Esse projeto prop\~{o}e um sistema para a realiza\c{c}\~{a}o de consultas OLAP em um Banco de Dados em Grafo sem a necessidade da gera\c{c}\~{a}o de um modelo intermedi\'{a}rio de dados para realizar an\'{a}lise em grafos heterog\^{e}neos. O sistema proposto \'{e} capaz de responder consultas OLAP a partir de grafos agregados extra\'{i}dos do grafo original, al\'{e}m de tamb\'{e}m realizar an\'{a}lises acerca da topologia do grafo. Neste trabalho s\~{a}o apresentados experimentos mostrando a efic\'{a}cia do sistema para responder \`{a}s consultas anal\'{i}ticas e compara\c{c}\~{o}es espec\'{i}ficas entre o sistema descrito e as solu\c{c}\~{o}es existentes.
% \noindent %- o resumo deve ter apenas 1 parágrafo e sem recuo de texto na primeira linha, essa tag remove o recuo. Não pode haver quebra de linha.

 \vspace{\onelineskip}

 \noindent
 \textbf{Palavras-chaves}: OLAP; bando de dados em grafo; grafos; an\'{a}lise de dados.
\end{resumo}
